\documentclass[11pt]{article}
\pdfoutput=1
\usepackage[T2A]{fontenc}
\usepackage[utf8]{inputenc}
\usepackage[english, russian]{babel}
\usepackage{NotesTeX_rus}
\usepackage{a4wide}
% Математика
\usepackage{amsmath,amsfonts,amssymb,amsthm,mathtools} 
\usepackage{wasysym}
\usepackage{indentfirst}
\usepackage{a4wide} 
\usepackage[T2A]{fontenc}			% кодировка
\definecolor{block-gray}{gray}{0.90} % уровень прозрачности (1 - максимум)
\newtcolorbox{myquote}{colback=red!5!white,grow to right by=0mm,grow to left by=0mm, boxrule=0pt,boxsep=0pt,breakable} % 1настройки области с изменённым фоном
%Заголовок
%\usepackage{graphix}
%зададим полезные команды:
%скалярное произведение
\newcommand{\scalar}[2]{\left \langle #1,#2\right \rangle}
%выпуклая оболочка
\newcommand{\diag}{\textrm{diag}}
\newcommand{\comp}{\textrm{comp}}
\newcommand{\conv }[0]{\textrm{conv}}
\newcommand{\sgn}[1]{\textrm{sgn}\left(#1\right)}
\newcommand{\Si}[1]{\textrm{Si}\left(#1\right)}

\begin{document}
\thispagestyle{empty}

\begin{center}
\ \vspace{-3cm}

\includegraphics[width=0.5\textwidth]{msu.eps}\\
{\scshape Московский государственный университет имени М.~В.~Ломоносова}\\
Факультет вычислительной математики и кибернетики\\
Кафедра системного анализа

\vfill

{\LARGE Курсовая работа}

\vspace{1cm}

{\LARGE\bfseries <<Динамические системы и модели биологии>>} \\
\vspace{1cm}
{\Large\bfseries <<Часть 2: Исследование нелинейных  \\
динамических систем на плоскости>>}
\end{center}

\vspace{1cm}

\begin{flushright}
  \large
  \textit{Студент 315 группы}\\
  И.\,В.~Шамков

  \vspace{5mm}

  \textit{Преподаватель}\\
  к.ф.-м.н., доцент И.\,В.~Востриков
\end{flushright}

\vfill

\begin{center}
Москва, 2022
\end{center}

\newpage
\tableofcontents
\newpage
\section{Постановка задачи}
Задана динамическая система
\begin{equation*}
\begin{cases}
\frac{du}{dt} = r_1u (A_u - u) - k_1uh, \\
\frac{dv}{dt} = r_2v (A_v - v) - k_2uh, \\
\frac{dh}{dt} = - \gamma h + R, \\
v(0) = 10, h(0) = 0, u(0) = 10^3
\end{cases}
(u,v,h) \in \mathbb{R}^3_+
\end{equation*}
$ r_1 = 0.012,r_2 = 0.006, A_u = 10^{12}, A_v = 10^{10}, k_2 = 10^{-6} 
  k_1 = 4.25 k_2,$ \\ $ \gamma = 0.0001 = 10^{-3}.$
Найти значение параметра $R$, при котором $v(t) >= 10^{5}$. \\
Необходимо:
\begin{enumerate}
    \item Дать биологическую интерпретацию характеристик системы.
    \item Ввести новые безразмерные переменные, максимально уменьшив число входящих параметров. Выбрать два свободных параметра. Если число параметров больше двух, то считать остальные параметры фиксированными.
    \item Найти неподвижные точки системы и исследовать их характер в зависимости от значений параметров. Результаты исследования представить в виде параметрического портрета системы.
    \item Для каждой характерной области параметрического портрета построить фазовый портрет. Дать характеристику поведения системы в каждом из этих случаев.
    \item Исследовать возможность возникновения предельного цикла. В положительном случае найти соответствующее первое ляпуновское число. Исследовать характер предельного цикла (устойчивый, неустойчивый, полуустойчивый).
    \item Дать биологическую интерпретацию полученным результатам.
\end{enumerate}
\section{Биологическая интерпретация}

\end{document}